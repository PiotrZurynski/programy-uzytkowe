\documentclass[10pt,a4paper]{article}
\usepackage[MeX]{polski}
\usepackage[utf8]{inputenc}

\begin{document}

$a$ do kwadratu plus~$b$ do kwadratu równa się~$c$ do kwadratu. 
Albo, stosując bardziej matematyczne podejście:$c^{2}=a^{2}+b^{2}$.

{\TeX} należy wymawiać jako $\tau\epsilon\chi$. \\[6pt]
100~m$^{3}$ wody. \\[6pt]
To płynie z~mojego~$\heartsuit$.

$a$ do kwadratu plus~$b$
do kwadratu równa się~$c$
do kwadratu. Albo, bardziej matematycznie 
\begin{displaymath}
c^{2}=a^{2}+b^{2}
\end{displaymath}
Pierwszy wiersz po wzorze

\begin{equation}
\epsilon > 0 \label{eq:eps}
\end{equation}
Ze wzoru (\ref{eq:eps})
otrzymujemy \ldots

$\lim_{n \to \infty}
\sum_{k=1}^n \frac{1}{k^2}
= \frac{\pi^2}{6}$

\begin{displaymath}
\lim_{n \to \infty}
\sum_{k=1}^n \frac{1}{k^2}
= \frac{\pi^2}{6}
\end{displaymath}

\begin{equation}
\forall x \in \mathbf{R}\colon
\qquad x^{2} \geq 0
\end{equation}

\begin{equation}
x^{2} \geq 0\qquad
\textrm{dla każdego }x\in\mathbf{R}
\end{equation}

\begin{displaymath}
x^{2} \geq 0\qquad
\textrm{dla każdego}x\in\mathbf{R}
\end{displaymath}

\begin{equation}
a^x+y \neq a^{x+y}
\end{equation}

$\lambda,\xi,\pi,\mu,\Phi,\Omega$

$a_{1} x^{2} e^{-\alpha t}
a^{3}_{ij} e^{x^2} \neq {e^x}^2$

$\sqrt{x} \sqrt{ x^{2}+\sqrt{y}}
\sqrt{3}{2} \surd[x^2 +y^2]$

$\overline{m+n} \underline{x+y}$

$\underbrace{ a+b+\cdots+z }_{26}$

\begin{displaymath}
\hat y=x^{2}\quad y'=2x'''
\end{displaymath}

\begin{displaymath}
\vec a\quad\overrightarrow{AB}
\end{displaymath}

\[\lim_{x \rightarrow 0}
\frac{\sin x}{x}=1\]

$a\bmod b$\\
$x\equiv a \pmod{b}$

$1\frac{1}{2}$~godziny
\begin{displaymath}
\frac{ x^{2} } { k+1 }\quad
x^{ \frac{2}{k+1} }\quad x^{ 1/2 }
\end{displaymath}

\begin{displaymath}
{n \choose k}\qquad {x \atop y+2}
\end{displaymath}

\begin{displaymath}
\int f_N(x) \stackrel{!}{=} 1
\end{displaymath}

\begin{displaymath}
\sum_{i=1}^n \quad
\int_{0}^{\frac{\pi}{2}}\qquad
\prod_\epsilon
\end{displaymath}

\begin{displaymath}
{a,b,c}\neq\{a,b,c\}
\end{displaymath}

\begin{displaymath}
1 + \left( \frac{1}{ 1-x^{2} }
\right)   ^3
\end{displaymath}

$\Big( (x+1) (x-1) \Big)^{2}$\\
$\big(\Big(\bigg(\Bigg($\quad
$\big\}\Big\}\bigg\}\Bigg\}$\quad
$\big\|\Big\|\bigg\|\Bigg\|$

\begin{displaymath}
x_{1},\ldots,x_{n} \qquad
x_{1}+\cdots+x_{n}
\end{displaymath}

\newcommand{\ud}{\mathrm{d}}
\begin{displaymath}
\int\!\!\!\int_{D} g(x,y)
\, \ud x\, \ud y
\end{displaymath}
%
zamiast 
\begin{displaymath}
\int\int{D} g(x,y)\ud x \ud y
\end{displaymath}

\begin{displaymath}
\mathbf{X} =
\left( \begin{array}{ccc}
x_{11} & x_{12} & \ldots \\
x_{21} & x_{22} & \ldots \\
\vdots & \vdots & \ldots
\end{array} \right)
\end{displaymath}

\begin{displaymath}
\mathbf{Y} =
\left( \begin{array}{ccc}
a & \textrm{jeżeli $d>c$}\\
b+x & \textrm{rano}\\
1 & \textrm{w~ciagu dnia}
\end{array} \right)
\end{displaymath}

\begin{displaymath}
\left(\begin{array}{c|c}
1 & 2 \\ \hline
3 & 4
\end{array}\right)
\end{displaymath}

\begin{eqnarray}
f(x) &  =  \cos x \\
f'(x) & = & -\sin x \\
\int_{0}^{x} f(y)dy &
= & \sin x
\end{eqnarray}

{\setlength\arraycolsep{2pt}
\begin{eqnarray}
\sin x & = & x -\frac{x^{3}}{3!}
+\frac{x^{5}}{5!}-{}
\nonumber\\
& & {}-\frac{x^{7}}{7!}+{}\cdots
\end{eqnarray}

\begin{displaymath}
{} ^ {12}_{\phantom{1}6}\textrm{C}
\qquad \textrm{versus} \qquad
{}^{12}_{6}\textrm{C}
\end{displaymath}

\begin{displaymath}
\Gamma_{ij}^{\phantom{ij}k}
\qquad \textrm{versus} \qquad
\Gamma_{ij}^k
\end{displaymath}

\begin{equation}
2^{\textrm{nd rd}^\textrm{th}} \quad
2^{\mathrm{nd rd}^\mathrm{th}} 
\end{equation}

\begin{displaymath}
\mathop{\mathrm{cov}}(X,Y)=
\frac{\displaystyle
\sum_{i=1}^n(x_1-\ x)
(y_1- y)}
{\displaystyle\biggl[
\sum_{i=1}^n(x_1- x)^2
\sum_{i=1}^n(y_1- y)^2
\biggr]^{1/2}}
\end{displaymath}

% definicje w~preambule
\newtheorem{twr}{Twierdzenie}
\newtheorem{lem}{Lemat}
% po \begin{document}
\begin{lem} Pierwszy
lemat\dots\label{lem:1} \end{lem}
\begin{twr}[Dyzma]
Przyjmując w~lemacie~\ref{lem:1}
że $\epsilon=0$\dots \end{twr}
\begin{lem} Trzeci lemat\end{lem}

\newtheorem{mur}{Murphy}[section]
\begin{mur} Jeżeli coś można wykonać na dwa lub więcej sposobów, przy czym jeden z~nich prowadzi do katastrofy, to sposób ten zostanie przez kogoś wybrany.
\end{mur}


\begin{proof}
Banalne. Użyj \[E=mc^2\]
\end{proof}


\end{document}